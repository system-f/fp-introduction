\begin{frame}
\begin{center}
What is Functional Programming?
\end{center}
\begin{center}
What does it \emph{mean}?
\end{center}
\end{frame}

\begin{frame}[fragile]
\begin{block}{suppose the following program}
\begin{lstlisting}[style=java]
int wibble(int a, int b) {
  counter = counter + 1;
  return (a + b) * 2;
}

/* arbitrary code */

blobble(wibble(x, y), wibble(x, y));
\end{lstlisting}
\end{block}
\end{frame}

\begin{frame}[fragile]
\begin{block}{and we refactor out these common expressions}
\begin{lstlisting}[style=java]
int wibble(int a, int b) {
  counter = counter + 1;
  return (a + b) * 2;
}

/* arbitrary code */

blobble(`wibble(x, y)`, `wibble(x, y)`);
\end{lstlisting}
\end{block}
\end{frame}

\begin{frame}[fragile]
\begin{block}{and we refactor out these common expressions}
\begin{lstlisting}[style=java]
int wibble(int a, int b) {
  counter = counter + 1;
  return (a + b) * 2;
}

int r = `wibble(x, y);`

/* arbitrary code */

blobble(`r`, `r`);
\end{lstlisting}
\end{block}
\end{frame}

\begin{frame}[fragile]
\begin{center}
Did the program just change?
\end{center}
\end{frame}

\begin{frame}[fragile]
\begin{block}{Yes, the program changed because \ldots}
\begin{lstlisting}[style=java]
int wibble(int a, int b) {
  `counter = counter + 1;`
  return (a + b) * 2;
}

int r = wibble(x, y);

/* arbitrary code */

blobble(r, r);
\end{lstlisting}
\end{block}
\end{frame}

\begin{frame}[fragile]
\begin{block}{suppose this slightly different program}
\begin{lstlisting}[style=java]
int pibble(int a, int b) {
  return (a + b) * 2;
}

/* arbitrary code */

globble(pibble(x, y), pibble(x, y));
\end{lstlisting}
\end{block}
\end{frame}

\begin{frame}[fragile]
\begin{block}{and we refactor out these common expressions}
\begin{lstlisting}[style=java]
int pibble(int a, int b) {
  return (a + b) * 2;
}

/* arbitrary code */

globble(`pibble(x, y)`, `pibble(x, y)`);
\end{lstlisting}
\end{block}
\end{frame}

\begin{frame}[fragile]
\begin{block}{and we refactor out these common expressions}
\begin{lstlisting}[style=java]
int pibble(int a, int b) {
  return (a + b) * 2;
}

int r = `pibble(x, y);`

/* arbitrary code */

globble(`r`, `r`);
\end{lstlisting}
\end{block}
\end{frame}

\begin{frame}[fragile]
\begin{center}
This time, did the program just change?
\end{center}
\end{frame}

\begin{frame}[fragile]
\begin{block}{it's the same program}
for given inputs, the same outputs are given, with no observable changes to the program
\end{block}
\end{frame}
% the definition
% the general consequences e.g. HOFs
% the concrete consequences e.g. no more loops
% the practical consequences e.g. reason about code, composition
